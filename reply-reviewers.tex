\documentclass[amsmath, preprintnumbers, 12pt, onecolumn, pre, longbibliograpy]{revtex4-1}
\usepackage{graphicx}
\usepackage[margin = 1in]{geometry}

\usepackage{subfigure}
\usepackage{color}
\newcommand{\MW}{\mathbf W}
\newcommand{\MA}{\mathbf A}
\newcommand{\MB}{\mathbf B}
\newcommand{\MX}{\mathbf X}
\newcommand{\aff}{\mathcal A}
\newcommand{\R}{\mathcal R}
\newcommand{\vP}{\mathbf P}
\renewcommand{\Re}{{\mathrm Re}}
\renewcommand{\Im}{{\mathrm Im}}
\usepackage{physics}


\begin{document}

\thispagestyle{empty}
\noindent Dear Referees,

\vspace{1cm}

Thank you for taking the time to carefully read our manuscript and for your thoughtful comments and criticisms. Please find below a summary of the revisions we have made, including revised figures, followed by a detailed response to each of your reviews.

\section*{Summary of changes}

\pagebreak

\section*{Point-by-point reply}

The referee's comments are copied in italics, with our responses below.

\subsection*{Referee 1}

\textit{ This is a paper of prime scientific quality that addresses the robustness of oscillations in cyclical reaction networks that are decorated by arbitrary side cycles. The work is solid and beyond scientific reproach (and hence I placed it as the ``top 5\%"), but there is an issue with the presentation: Do the authors really think JCP is the right venue for this article? It is way too specialized to be understood by even a small fraction of JCP readers. I have 3 decades of experience as a quantum chemical physics scientists and I had tremendous difficulty reading the paper due to the very specialized jargon (had to read certain passages times and time again, and to make things worse, many symbols in equations are not even defined in the text (some are defined only in figure captions) - to fully appreciate the paper one has to really read the previous paper from the authors' lab (and I did not have the time, unfortunately, to do so). A concrete application to a biochemical cycle would also be highly recommended.}

\subsection*{Referee 2}

\textit{In ``Robust oscillations in multi-cyclic models of biochemical clocks" the authors analyze the number and period of coherent oscillations in a continuous-time Markov chain without detailed balance. They build on a previous recent work where analytical expressions for these quantities where derived in the case of a unicycle chain very far from equilibrium. In the present manuscript they study the modifications induced by adding side cycles-on top of randomizing the transition rates-applying a coarse-graining procedure that leads them back to the unicycle case. The work is interesting, and well-written in places, and the theoretical/numerical results look solid. Nevertheless, I would appreciate the authors to address the following questions to help me pronouncing on the overall validity and relevance of the manuscript.}

\textit{i) On the basis of Eqs (4)-(6) - already derived in a preceding work and valid at large affinities-one clearly understands that ``the rates decouple from each other and only contribute additively to the timescales". The authors claim that, as a result, "the farther an oscillator operates from equilibrium, the more robust it will be to fluctuations in the rates ". I do not see how this last statement follows from the previous one. It would do so, if they showed that the coupling between the heterogenous rates always increases in modulus $C \gamma$. Can this be seen from the derivation of Eqs (4)-(6)?}

\textit{ii) The authors refer several time to the robustness of oscillations without defining it precisely. This leaves room to objections to their statement that higher affinity brings higher robustness. For example, we could require that the number of coherent oscillations $R$ decreases the least possible with respect to its value $R_0$ in the homogeneous (unicycle) case. If so, from Fig. 3 (lower left panel) we see that the lower affinity gives the highest $R/R_0$, i.e. it is actually the most robust condition. The same can be concluded looking at the exact values of $R$ in Fig. 4.
The authors should thus specify which notion of robustness they are considering to make their statement sound.}

\textit{iii) The terminology employed in the conclusion is somewhat unclear. A net variation in the affinity is called ``long-lived, global rate fluctuations". However, the ``quenched disorder" in the rates could already by described as `long-lived and global'. The authors maybe mean that a variation in affinity represents extrinsic or environmental noise, while 'local' inhomogeneities in the rates effectively describe intrinsic noise?}

\textit{iv) Clearly, a stochastic chemical reaction network is different from the Markov chain considered by the authors. In particular, since the dynamics is ruled by the Chemical Master Equation, the transition rates (with e.g. mass action kinetics) are state dependent, in general, and not constant.
Therefore, the present model should be seen as an abstract one to describe chemistry. In which sense is this an effective description? What kind of relation is there between the real chemical affinity (i.e. the difference in nonequilibrium free energy) and the present affinity?
All in all, should the effects observed for this Markov chain carry over to chemical reaction networks with mass action kinetics?}


\textit{Typos:
"...R and T change by only a small fraction as ? in turned on..." -> "...is turned on..."
"...high energy dissipation makes these observables insensitive tomany parameters" -> "...to many..."}



\vspace{2cm}
\noindent Regards,
\vspace{1cm}

\noindent Suriyanarayanan Vaikuntanathan (Corresponding Author, Email: \href{mailto:svaikunt@uchicago.edu}{svaikunt@uchicago.edu})

\noindent Clara del Junco


%\bibliography{/Users/claradeljuncooffice/Documents/library.bib}

\end{document}