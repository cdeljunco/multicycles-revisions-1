\documentclass[amsmath, preprintnumbers, 10pt, twocolumn, pre, bibliograpy]{revtex4-1}
%DIF LATEXDIFF DIFFERENCE FILE
%DIF DEL ../V1/main.tex   Wed Oct  9 13:56:16 2019
%DIF ADD main.tex         Wed Dec 18 17:50:53 2019
%\documentclass[amsmath, preprint, 12pt, onecolumn, pre, longbibliograpy, notitlepage]{revtex4-1}

\usepackage{graphicx}
\usepackage{subfigure}
\usepackage{color}
\usepackage{amsfonts}
\usepackage{amsmath}
\usepackage{bbm}
\newcommand{\MW}{\mathbf W}
\newcommand{\MA}{\mathbf A}
\newcommand{\MB}{\mathbf B}
\newcommand{\MX}{\mathbf X}
\newcommand{\aff}{\mathcal A}
\newcommand{\R}{\mathcal R}
\newcommand{\vP}{\mathbf P}
\renewcommand{\Re}{{\mathrm Re}}
\renewcommand{\Im}{{\mathrm Im}}
\usepackage{physics}
\newcommand{\mfpt}{\langle \tau \rangle}
%DIF PREAMBLE EXTENSION ADDED BY LATEXDIFF
%DIF UNDERLINE PREAMBLE %DIF PREAMBLE
\RequirePackage[normalem]{ulem} %DIF PREAMBLE
\RequirePackage{color}\definecolor{RED}{rgb}{1,0,0}\definecolor{BLUE}{rgb}{0,0,1} %DIF PREAMBLE
\providecommand{\DIFadd}[1]{{\protect\color{blue}\uwave{#1}}} %DIF PREAMBLE
\providecommand{\DIFdel}[1]{{\protect\color{red}\sout{#1}}}                      %DIF PREAMBLE
%DIF SAFE PREAMBLE %DIF PREAMBLE
\providecommand{\DIFaddbegin}{} %DIF PREAMBLE
\providecommand{\DIFaddend}{} %DIF PREAMBLE
\providecommand{\DIFdelbegin}{} %DIF PREAMBLE
\providecommand{\DIFdelend}{} %DIF PREAMBLE
%DIF FLOATSAFE PREAMBLE %DIF PREAMBLE
\providecommand{\DIFaddFL}[1]{\DIFadd{#1}} %DIF PREAMBLE
\providecommand{\DIFdelFL}[1]{\DIFdel{#1}} %DIF PREAMBLE
\providecommand{\DIFaddbeginFL}{} %DIF PREAMBLE
\providecommand{\DIFaddendFL}{} %DIF PREAMBLE
\providecommand{\DIFdelbeginFL}{} %DIF PREAMBLE
\providecommand{\DIFdelendFL}{} %DIF PREAMBLE
\newcommand{\DIFscaledelfig}{0.5}
%DIF HIGHLIGHTGRAPHICS PREAMBLE %DIF PREAMBLE
\RequirePackage{settobox} %DIF PREAMBLE
\RequirePackage{letltxmacro} %DIF PREAMBLE
\newsavebox{\DIFdelgraphicsbox} %DIF PREAMBLE
\newlength{\DIFdelgraphicswidth} %DIF PREAMBLE
\newlength{\DIFdelgraphicsheight} %DIF PREAMBLE
% store original definition of \includegraphics %DIF PREAMBLE
\LetLtxMacro{\DIFOincludegraphics}{\includegraphics} %DIF PREAMBLE
\newcommand{\DIFaddincludegraphics}[2][]{{\color{blue}\fbox{\DIFOincludegraphics[#1]{#2}}}} %DIF PREAMBLE
\newcommand{\DIFdelincludegraphics}[2][]{% %DIF PREAMBLE
\sbox{\DIFdelgraphicsbox}{\DIFOincludegraphics[#1]{#2}}% %DIF PREAMBLE
\settoboxwidth{\DIFdelgraphicswidth}{\DIFdelgraphicsbox} %DIF PREAMBLE
\settoboxtotalheight{\DIFdelgraphicsheight}{\DIFdelgraphicsbox} %DIF PREAMBLE
\scalebox{\DIFscaledelfig}{% %DIF PREAMBLE
\parbox[b]{\DIFdelgraphicswidth}{\usebox{\DIFdelgraphicsbox}\\[-\baselineskip] \rule{\DIFdelgraphicswidth}{0em}}\llap{\resizebox{\DIFdelgraphicswidth}{\DIFdelgraphicsheight}{% %DIF PREAMBLE
\setlength{\unitlength}{\DIFdelgraphicswidth}% %DIF PREAMBLE
\begin{picture}(1,1)% %DIF PREAMBLE
\thicklines\linethickness{2pt} %DIF PREAMBLE
{\color[rgb]{1,0,0}\put(0,0){\framebox(1,1){}}}% %DIF PREAMBLE
{\color[rgb]{1,0,0}\put(0,0){\line( 1,1){1}}}% %DIF PREAMBLE
{\color[rgb]{1,0,0}\put(0,1){\line(1,-1){1}}}% %DIF PREAMBLE
\end{picture}% %DIF PREAMBLE
}\hspace*{3pt}}} %DIF PREAMBLE
} %DIF PREAMBLE
\LetLtxMacro{\DIFOaddbegin}{\DIFaddbegin} %DIF PREAMBLE
\LetLtxMacro{\DIFOaddend}{\DIFaddend} %DIF PREAMBLE
\LetLtxMacro{\DIFOdelbegin}{\DIFdelbegin} %DIF PREAMBLE
\LetLtxMacro{\DIFOdelend}{\DIFdelend} %DIF PREAMBLE
\DeclareRobustCommand{\DIFaddbegin}{\DIFOaddbegin \let\includegraphics\DIFaddincludegraphics} %DIF PREAMBLE
\DeclareRobustCommand{\DIFaddend}{\DIFOaddend \let\includegraphics\DIFOincludegraphics} %DIF PREAMBLE
\DeclareRobustCommand{\DIFdelbegin}{\DIFOdelbegin \let\includegraphics\DIFdelincludegraphics} %DIF PREAMBLE
\DeclareRobustCommand{\DIFdelend}{\DIFOaddend \let\includegraphics\DIFOincludegraphics} %DIF PREAMBLE
\LetLtxMacro{\DIFOaddbeginFL}{\DIFaddbeginFL} %DIF PREAMBLE
\LetLtxMacro{\DIFOaddendFL}{\DIFaddendFL} %DIF PREAMBLE
\LetLtxMacro{\DIFOdelbeginFL}{\DIFdelbeginFL} %DIF PREAMBLE
\LetLtxMacro{\DIFOdelendFL}{\DIFdelendFL} %DIF PREAMBLE
\DeclareRobustCommand{\DIFaddbeginFL}{\DIFOaddbeginFL \let\includegraphics\DIFaddincludegraphics} %DIF PREAMBLE
\DeclareRobustCommand{\DIFaddendFL}{\DIFOaddendFL \let\includegraphics\DIFOincludegraphics} %DIF PREAMBLE
\DeclareRobustCommand{\DIFdelbeginFL}{\DIFOdelbeginFL \let\includegraphics\DIFdelincludegraphics} %DIF PREAMBLE
\DeclareRobustCommand{\DIFdelendFL}{\DIFOaddendFL \let\includegraphics\DIFOincludegraphics} %DIF PREAMBLE
%DIF END PREAMBLE EXTENSION ADDED BY LATEXDIFF

\begin{document}
\title{Robust oscillations in multi-cyclic \DIFaddbegin \DIFadd{Markov state }\DIFaddend models of biochemical clocks}
\author{Clara \surname{del Junco} and Suriyanarayanan Vaikuntanathan}
\email{Corresponding Author: svaikunt@uchicago.edu}
\affiliation{Department of Chemistry and The James Franck Institute, University of Chicago, Chicago, IL, 60637}


\begin{abstract}
\DIFaddbegin \DIFadd{Organisms often use cyclic changes in the concentration of chemicals species to precisely biological functions. Underlying these biochemical clocks are series of chemical reactions and transport processes, which are inherently stochastic in the small and crowded environment of a cell. }\DIFaddend Understanding the physical basis for robust biochemical oscillations in the presence of fluctuations, as observed for instance in \DIFaddbegin \DIFadd{experiments of simple }\DIFaddend circadian oscillators, has emerged as an important problem. In a previous paper (arXiv:1808.04914), we explored this question using the non-equilibrium statistical mechanics of single-ring Markov state models of biochemical networks that support oscillations. Our finding was that they can exploit \DIFdelbegin \DIFdel{high affinity }\DIFdelend \DIFaddbegin \DIFadd{nonequilibrium driving }\DIFaddend to maintain a robust period and coherence in the presence of fluctuations in rates. Here, we extend our work to Markov state models consisting of a large cycle decorated with multiple small cycles. These additional cycles are intended to represent alternate pathways that the oscillator may take as it fluctuates about its average path. Combining a mapping to single-cycle networks based on first passage time distributions with \DIFaddbegin \DIFadd{our }\DIFaddend previously developed theory, we are able to make analytical predictions for the period and coherence of oscillations in these networks. One implication of our predictions is that for these networks\DIFaddbegin \DIFadd{, }\DIFaddend a high energy budget can make different network topologies and arrangements of rates degenerate as far as the period and coherence of oscillations is concerned. Excellent agreement between analytical and numerical results confirms that this is the case. Our results \DIFdelbegin \DIFdel{suggest }\DIFdelend that biochemical oscillators can be more robust to fluctuations in the path of the oscillator when they have a high energy budget.
\end{abstract}

\maketitle 

\section{Introduction} %draft done

Many organisms use internal biochemical clocks to synchronize their metabolism to day-night cycles\DIFdelbegin \DIFdel{. An internal record of time has been shown to confer fitness to organisms }\DIFdelend \DIFaddbegin \DIFadd{, a tactic that confers fitness }\DIFaddend as it allows them to anticipate periodic environmental changes~\cite{Woelfle2004}. \DIFaddbegin \DIFadd{These clocks are implemented as a series of chemical reactions and transport processes, whose timing can be affected by intrinsic and extrinsic noise - yet typically these clocks have evolved to maintain consistent periods over different copies of the oscillator (e.g. in different cells), and over time. For example, the core clocks proteins of the circadian oscillator of }{\it \DIFadd{S. Elongatus}} \DIFadd{bacteria, generate oscillations in KaiC phosphorylation level with a 24-hour period over many days even in constant light or dark conditions~\mbox{%DIFAUXCMD
\cite{Nakajima2005, Tomita2005, Rust2007}}%DIFAUXCMD
.  }\DIFaddend From a theoretical standpoint, understanding the non-equilibrium statistical mechanical requirements for maintaining robust oscillations in these molecular clocks has \DIFaddbegin \DIFadd{thus }\DIFaddend emerged as an important question. In particular, the connection between energy dissipation and the \DIFdelbegin \DIFdel{coherence of oscillations in a stochastic environment }\DIFdelend \DIFaddbegin \DIFadd{precision of the stochastic period of oscillators }\DIFaddend has been noted by many theoretical studies~\cite{Barato2015, Cao2015, Barato2017, Fei2018, Wierenga2018, Nguyen2018, Marsland2019}. Some of this work proposes thermodynamic bounds which set a lower limit on the extent of stochastic fluctuations in these systems as a function of the energy dissipation budget~\cite{Barato2015, Barato2017, Wierenga2018}.  However, the structure of even simple models of biochemical oscillators constrains them to operate far from these bounds~\cite{Marsland2019}\DIFdelbegin \DIFdel{- i.e., }\DIFdelend \DIFaddbegin \DIFadd{, which raises the question of what role energy dissipation plays in these cases where }\DIFaddend fluctuations are much larger than the minimum for the amount of energy that the oscillator is using.
\DIFdelbegin \DIFdel{In particular, fluctuations are minimized only in the case of a symmetric oscillator, that is, when all of the clockwise rates $k^+_i$ in the model in Fig.~\ref{fig:schematic}a are equal to one another and related to the counterclockwise rates $k^-_i$ by $k^+ = \exp(\aff/N) k^-$ where $\aff$ is the affinity, the non-equilibrium driving force that drives the system out of equilibrium which is typically provided by ATP hydrolysis, and $N$ is the size of the cycle. %DIF < This raises the question of the how the energy budget is allocated in these systems, and whether it plays a role in controlling the effects of fluctuations other than those considered in this existing body of work.  
}\DIFdelend 

\DIFdelbegin %DIFDELCMD < \begin{figure}
%DIFDELCMD < \centering
%DIFDELCMD < \includegraphics[width=\linewidth]{fig-1.pdf}
%DIFDELCMD <  %%%
%DIFDELCMD < \caption{%
{%DIFAUXCMD
\DIFdelFL{A schematic of the networks studied in this article. (a) In previous work~\mbox{%DIFAUXCMD
\cite{DelJunco2018b}}%DIFAUXCMD
, we developed an analytical theory for the period of oscillations $T$ and the number of coherent oscillations $\R$, defined in Eq.~\ref{eq:r-t}, in oscillators that can be represented by a single cycle of states where the clockwise (CW) hopping rates ($k_i^+$) are much larger than the counterclockwise (CCW) hopping rates ($k_i^-$). This asymmetry is quantified by the affinity $\mathcal A$, defined in Eq.~\ref{eq:aff}. (b) In this work, we extend our results to networks where the main cycle is decorated with many small cycles. (c) We design these `decorations' so that the rates going in to them are modulated by a small parameter $\mu$ which governs the probability that the system will enter the small cycle. (d) To apply our theory to these multi-cycle networks, we map the muti-cyclic network on to a unicyclic network by matching moments of the first passage time distribution for a random walk beginning CCW from the decoration (green circle) and ending CW from it (red circle) on to a line of states with two unknown hopping rate. This procedure gives a set of effective hopping rates $\eta_i^\pm$. When each of the decorations in the network is coarse-grained in this manner, we obtain a single-cycle network as in (a) whose period $T$ and coherence $\R$ should approximate those of the full, multicycle network.}}
%DIFAUXCMD
%DIFDELCMD < \label{fig:schematic}
%DIFDELCMD < \end{figure}
%DIFDELCMD < 

%DIFDELCMD < %%%
\DIFdelend Recently, we \DIFdelbegin \DIFdel{considered how a }\DIFdelend \DIFaddbegin \DIFadd{explored this question using }\DIFaddend single-cycle Markov \DIFdelbegin \DIFdel{model of an oscillator }\DIFdelend \DIFaddbegin \DIFadd{models of biochemical oscillators }\DIFaddend such as that pictured in Fig.~\ref{fig:schematic}a\DIFdelbegin \DIFdel{operates when the rates are not symmetric, but rather chosen from a probability and subject to fluctuations, as is reasonably expected in any chemical reaction network operating in a noisy biological environment~\mbox{%DIFAUXCMD
\cite{DelJunco2018b}}%DIFAUXCMD
. }\DIFdelend \DIFaddbegin \DIFadd{. %DIF > operates when the rates are not symmetric, but rather chosen from a probability and subject to fluctuations, as is reasonably expected in any chemical reaction network operating in a noisy biological environment~\cite{DelJunco2018b}. 
}\DIFaddend By deriving an analytical expression for the period and coherence of oscillations that reveals their detailed dependence on all of the rates in the network, we showed that \DIFdelbegin \DIFdel{in addition to suppressing fluctuations due to the inherent stochasticity of the Poisson hopping events }\DIFdelend \DIFaddbegin \DIFadd{nonequilibrium driving allows the period of oscillators to become insensitive to many of the parameters of the models - specifically, the arrangement of the transition rates }\DIFaddend on the ring\DIFdelbegin \DIFdel{and the possibility of reverse hops, a high chemical affinity, defined as 
}\begin{displaymath}
\DIFdel{\aff = \sum_{cycle} k_i^{+}/k_i^{-},
%DIFDELCMD < \label{eq:aff}%%%
}\end{displaymath}
%DIFAUXCMD
\DIFdelend \DIFaddbegin \DIFadd{. Driving thus }\DIFaddend allows the period of \DIFdelbegin \DIFdel{the oscillator to remain more tightly distributed when the rates in the model fluctuate}\DIFdelend \DIFaddbegin \DIFadd{a wide class of oscillators - even those operating far from the bound - to be robustly maintained in the presence of changes in these parameters}\DIFaddend . In this paper, we \DIFdelbegin \DIFdel{build on our previous work to }\DIFdelend further explore this role of energy dissipation \DIFdelbegin \DIFdel{. Specifically, we extend our work }\DIFdelend \DIFaddbegin \DIFadd{by extending our results }\DIFaddend to networks with multiple cycles\DIFdelbegin \DIFdel{and }\DIFdelend \DIFaddbegin \DIFadd{. We }\DIFaddend show that the period of the oscillator is more robust at high \DIFdelbegin \DIFdel{affinity when the }\DIFdelend \DIFaddbegin \DIFadd{driving to changes in the the }\DIFaddend topology of the network \DIFdelbegin \DIFdel{fluctuates }\DIFdelend as well as the rates.

The paper is organized as follows: in Section\DIFaddbegin \DIFadd{~\ref{model} we introduce the class of multicyclic Markov state models considered in this paper and define nonequilibrium driving, the observables of interest (period and coherence of oscillations), and robustness in the context of these models. In Section }\DIFaddend \ref{theory} we briefly review our analytical theory from Ref.~\citenum{DelJunco2018b}. This theory depends on the single-cycle topology of the \DIFdelbegin \DIFdel{network}\DIFdelend \DIFaddbegin \DIFadd{networks studied in Ref.~\mbox{%DIFAUXCMD
\citenum{DelJunco2018b}}%DIFAUXCMD
}\DIFaddend , so to apply it to multi-cycle networks such at the one illustrated in Fig.~\ref{fig:schematic}b, in Section~\ref{CG} we show how to coarse-grain small cycles on to single links, yielding an effective single-cycle network whose period and coherence are meant to approximate those of the full multi-cyclic network. In Section~\ref{timescales} we compare these observables calculated numerically in multi-cycle networks to the corresponding coarse-grained networks and analytical approximations, and show that at high affinity our analytical approximation, which takes as input only a small subset of the parameters required to specify the multi-cycle network, accurately reproduces the period and coherence in these networks. We demonstrate the ability of our theory to predict timescales when the rates and topology in the network are randomly generated. Finally, in Section~\ref{disc}, we discuss the implications of our results for biochemical oscillators and show one example of how the multi-cycle networks studied here can be used to achieve input compensation, which is the ability to maintain a constant period when the affinity changes.

\DIFdelbegin \section{\DIFdel{Analytical theory for timescales in single-cycle oscillator models}}%DIFAUXCMD
\addtocounter{section}{-1}%DIFAUXCMD
%DIFDELCMD < \label{theory} %%%
%DIF < draft done
\DIFdelend \DIFaddbegin \begin{figure}
\centering
\includegraphics[width=\linewidth]{fig-1.pdf}
 \caption{\DIFaddFL{A schematic of the networks studied in this article. (a) In previous work~\mbox{%DIFAUXCMD
\cite{DelJunco2018b}}%DIFAUXCMD
, we developed an analytical theory for the period of oscillations $T$ and the number of coherent oscillations $\R$, defined in Eq.~\ref{eq:r-t}, in oscillators that can be represented by a single cycle of states where the clockwise (CW) hopping rates ($k_i^+$) are much larger than the counterclockwise (CCW) hopping rates ($k_i^-$). This asymmetry is quantified by the affinity $\mathcal A$, defined in Eq.~\ref{eq:aff}. (b) In this work, we extend our results to networks where the main cycle is decorated with many small cycles. (c) We design these `decorations' so that the rates going into them are modulated by a small parameter $\mu$ which governs the probability that the system will enter the small cycle. (d) To apply our theory to these multi-cycle networks, we map the muti-cyclic network on to a unicyclic network by matching moments of the first passage time distribution for a random walk beginning CCW from the decoration (green circle) and ending CW from it (red circle) on to a line of states with two unknown hopping rate. This procedure gives a set of effective hopping rates $\eta_i^\pm$. When each of the decorations in the network is coarse-grained in this manner, we obtain a single-cycle network as in (a) whose period $T$ and coherence $\R$ should approximate those of the full, multicycle network.}}
\label{fig:schematic}
\end{figure}
\DIFaddend 


\DIFdelbegin \DIFdel{In Ref.~\mbox{%DIFAUXCMD
\citenum{DelJunco2018b}}%DIFAUXCMD
, we derived analytical expressions for the period and coherence of oscillations in a network consisting of a single cycle of states, such as the one depicted }\DIFdelend \DIFaddbegin \section{\DIFadd{Markov state models of biochemical oscillators}}\label{model}

\DIFadd{In this paper we consider Markov models such as the one in Fig.~\ref{fig:schematic} as simple models that capture the cycling and stochasticity of biochemical oscillators~\mbox{%DIFAUXCMD
\cite{Barato2017}}%DIFAUXCMD
.  In this model, each vertex represents the collective state of a system. For instance, in the KaiABC oscillator, it could be a vector of the counts of each phosphorylation state of KaiC~\mbox{%DIFAUXCMD
\cite{Rust2007}}%DIFAUXCMD
. The rates along each of the edges represent the rates of elementary processes, like a phosphorylation event. We emphasize that this picture is thus not a representation of the underlying chemical reaction network which must contain, at a minimum, a negative feedback loop, and may also have other motifs~\mbox{%DIFAUXCMD
\cite{Novak2008}}%DIFAUXCMD
. Rather, it is an emergent picture that captures the oscillations that can arise from such a network, and the feedback as well as mass action kinetics are is encoded in the rates along each edge, which depend on the collective state of the system represented by the connected vertices. It is not expected that a real oscillator will always follow the same path through its state space on each cycle. The single-cycle model }\DIFaddend in Fig.~\ref{fig:schematic}a \DIFdelbegin \DIFdel{. In this section, we briefly review that result. Further details are available in Ref.~\mbox{%DIFAUXCMD
\citenum{DelJunco2018b}}%DIFAUXCMD
. 
}\DIFdelend \DIFaddbegin \DIFadd{is a caricature that captures the average limit cycle of the oscillator. The multicycle model in Fig.~\ref{fig:schematic}b is a caricature intended to reflect of small fluctuations about this average. 
}

\DIFaddend The network is driven out of equilibrium by an affinity, $\mathcal A$, defined \DIFdelbegin \DIFdel{in Eq.~\ref{eq:aff} - a }\DIFdelend \DIFaddbegin \DIFadd{as 
}\begin{equation}
\DIFadd{\aff = \sum_{cycle} k_i^{+}/k_i^{-}.
\label{eq:aff}
}\end{equation}
\DIFadd{A }\DIFaddend finite affinity is necessary to have oscillations. The \DIFaddbegin \DIFadd{affinity is formally defined on closed cycles. In this work we refer to the affinity per site, $\aff/N = \langle k_i^{+}/k_i^{-} \rangle_{cycle}$, as a measure of the strength of driving. In biology, this nonequilibrium driving is typically provided by ATP hydrolysis. The }\DIFaddend quantities of interest are two time scales: the average period of oscillations $T$ and the number of coherent oscillations, $\R$, defined as
\begin{align}
T = 2\pi/ |\Im[\phi]| && \R = -|\Im[\phi]|/\Re[\phi] 
\label{eq:r-t}
\end{align}
where $\phi$ is the eigenvalue of the transition rate matrix of the network that yields the largest value of $\R$\DIFdelbegin \DIFdel{~}\DIFdelend \DIFaddbegin \DIFadd{. }\DIFaddend \footnote{In refs.~\citenum{DelJunco2018b} and \citenum{Barato2017}, $\phi$ was defined as the eigenvalue with the least negative real part. However, in the multicyclic networks we address later in this paper, that definition can lead to selecting an eigenvalue which corresponds to cycling around a decoration rather than global oscillations.}. \DIFaddbegin \DIFadd{Loosely speaking, a higher value of $\R$ corresponds to smaller fluctuations in the period. Although $\R$ is one measure of the quality of timekeeping in the clock, it is not the definition of robustness that we use in this paper. In \mbox{%DIFAUXCMD
\citenum{Barato2017}}%DIFAUXCMD
, it was postulated that $\mathcal R$ is maximized in a uniform oscillator, that is, when all of the clockwise rates $k^+_i$ in the model in Fig.~\ref{fig:schematic}a are equal to one another and related to the counterclockwise rates $k^-_i$ by $k^+ = \exp(\aff/N) k^-$. However, since the rates along each edge of the network depend on the collective state of the system represented by the connected vertices, a network that sustains oscillations cannot be uniform~\mbox{%DIFAUXCMD
\cite{Marsland2019}}%DIFAUXCMD
. Moreover, the rates and, in the case of the multicyclic network depicted in Fig.\ref{fig:schematic}b, the locations and size of the small secondary cycles, can fluctuate over time and between copies of the oscillator. In this paper we therefore consider two properties of the oscillator: first, how predictable the period of oscillations $T$ and the coherence $\R$ are with limited knowledge about the specific details of the arrangement of rates and decorations in the network, and by extension, how robust these quantities are to fluctuations in these details.
}

\section{\DIFadd{Analytical theory for timescales in single-cycle oscillator models}}\label{theory} %DIF > draft done

\DIFadd{In Ref.~\mbox{%DIFAUXCMD
\citenum{DelJunco2018b}}%DIFAUXCMD
, we derived analytical expressions for $\phi$, and therefore the period and coherence of oscillations, in a network consisting of a single cycle of states, as depicted in Fig.~\ref{fig:schematic}a. In this section, we briefly review that result. Further details are available in Ref.~\mbox{%DIFAUXCMD
\citenum{DelJunco2018b}}%DIFAUXCMD
. }\DIFaddend An exact expression for $\phi$ in a cycle of $N$ states in the special symmetric case where $k_i^{+} = k^+$ and $k_i^{-} = k^-$ for all $i$ is given by:
\begin{equation}
 \phi^{(0)} = -(k^- + k^+) + k^-e^{-2\pi i/N} + k^+e^{2\pi i/N} \label{eq:phi0}.
\end{equation}

We then considered networks where at least one of the CW \DIFaddbegin \DIFadd{(clockwise) }\DIFaddend rates is equal to $k^+$ and at least one of the CCW \DIFaddbegin \DIFadd{(counterclockwise) }\DIFaddend rates is equal to $k^-$. The remaining $m \leq N-1$ rates, denoted $h_j^\pm$, can be selected randomly from a distribution ranging over at least an order of magnitude above or below these ``uniform" rates. The main result of Ref.~\citenum{DelJunco2018b} was an expression for $\phi$ in this setup, in the limit of high affinity where terms of order $k^-/k^+ = \exp(-\aff/N)$ can be neglected compared to terms of order 1. The result is summarized in the following expressions: %which are the same as in Ref.~\citenum{DelJunco2018b} except for Eq.~\ref{eq:zeta} which has been modified to  relax some of the assumptions in Ref.~\citenum{DelJunco2018b} (further details in Appendix~\ref{app:theory}):
\begin{widetext}
\begin{align}
\phi &= \phi^{(0)} + C \gamma \label{eq:phinew} \\
 \phi^{(0)} & = -(k^- + k^+) + k^-\exp(-2 \pi i/N) + k^+\exp(2 \pi i/N) \\
\gamma &= \frac{1}{m-N}\left(\sum_{j=1}^m \log(\zeta_j(\gamma, k^\pm, h_j^\pm, N))\right) + \frac{1}{2(m-N)^2}\left(\sum_{j=1}^m \log(\zeta_j(\gamma, k^\pm, h_j^\pm, N))\right)^2 \label{eq:gamma} \\
%\zeta_j &=  \frac{{h_j^-} {k^+}+{h_j^+} {k^-}-{k^-} {k^+} + 2 \gamma  {h_j^+} {k^-} +\gamma ^2 \left({h_j^+} {k^-} + {k^-} {k^+} \right) + (\gamma +1) {k^+} e^{\frac{2 i \pi }{N}} (-{h_j^-}-{h_j^+}+{k^-}+{k^+})-\left((\gamma +1) {k^+} e^{\frac{2 i \pi }{N}}\right)^2}{(\gamma +1) {h_j^+} \left({k^-}-{k^+} e^{\frac{4 i \pi }{N}}\right)} \label{eq:zeta} \\
%C &=  c_1^2 k^- e^{-2\pi i/N} \label{eq:c} \\
%c_1^2 & = 1-(k^+/k^-)e^{4\pi i/N}. \label{eq:c1sq}
\end{align}
\end{widetext}
with expressions for $\zeta_j(\gamma, k^\pm, h_j^\pm, N))$ and $C$ given in Appendix~\ref{app:theory}. The essential feature of these equations is that Eq.~\ref{eq:gamma} depends independently on each rate $h_j^\pm$ and does not contain any information about the relative positions of the rates in the network. As a result, in the limit of moderately high affinity, fluctuations in the rates decouple from each other and only contribute additively to the timescales, and we find that $\R$ and $T$ are insensitive to the arrangement of the rates in the network.  From a biological perspective, this means that the farther an oscillator operates from equilibrium, the more robust it will be to fluctuations in the rates.

\DIFdelbegin \DIFdel{We note that }\DIFdelend \DIFaddbegin \DIFadd{Since }\DIFaddend Eq.~\ref{eq:gamma} is a self-consistent equation for $\gamma$\DIFdelbegin \DIFdel{. The }\DIFdelend \DIFaddbegin \DIFadd{, the }\DIFaddend theoretical predictions in the following section are obtained numerically by searching for solutions to Eq.~\ref{eq:gamma} near to a linear approximation of Eq.~\ref{eq:gamma}.

\section{Mapping multi-cycle to single-cycle oscillators via first passage time distributions}\label{CG} %draft done

\DIFaddbegin \DIFadd{We now wish to apply Eqs.~\ref{eq:phinew} - \ref{eq:gamma} to the networks with multiple cycles depicted in Fig.~\ref{schematic}b in order to understand how our conclusions extend to these higher-dimensional cases. }\DIFaddend The derivation of Eqs.~\ref{eq:phinew} - \ref{eq:gamma} used a transfer matrix technique which depended on the single-ring topology of the network. Rather than trying to extend this approach to networks with decorations of the kind we wish to consider here, depicted in Fig.~\ref{fig:schematic}a, we took a different approach and chose instead to map multi-cycle networks on to single-cycle networks so that Eqs.~\ref{eq:phinew}-\ref{eq:gamma} can then be directly applied to the mapped network. Because we want a mapping that preserves time scales, our approach is to build a single-cycle network with rates such that the mean and variance of the first passage time from a site upstream (in the sense of the probability current) of a \DIFdelbegin \DIFdel{decoration }\DIFdelend \DIFaddbegin \DIFadd{small cycle, which we call a decoration, }\DIFaddend to a site downstream of \DIFdelbegin \DIFdel{a cycle }\DIFdelend \DIFaddbegin \DIFadd{the decoration }\DIFaddend is preserved (denoted by the green and red circles in Fig.~\ref{fig:schematic}b). For each decoration, we replace the rates along the edge shared by the large and small cycles with an effective CW rate $\eta^+$ and an effective CCW rate $\eta^-$. 

In order to do this, we first calculate the first passage time (FPT) distribution across the decoration \DIFdelbegin \DIFdel{in the full network in }\DIFdelend \DIFaddbegin \DIFadd{(from the green circle to the red circle in Fig.~\ref{fig:schematic}b) in }\DIFaddend Laplace space~\cite{Murugan2012, Budnar2019} (details in Appendix \ref{app:fpt}).\DIFdelbegin \DIFdel{Though the }\DIFdelend \DIFaddbegin \DIFadd{The }\DIFaddend Laplace-transformed FPT distribution $\tilde F(s)$ \DIFdelbegin \DIFdel{may not be easy to invert to obtain the real-time FPT distribution $F(t)$, $\tilde F(s)$ }\DIFdelend is a moment-generating function for $F(t)$, with the $n$th moment $\langle \tau^n \rangle$ given by:
\begin{equation}
(-1)^n\left(\frac{\partial^n \hat F}{\partial s^n}\right)_{s = 0} = \langle \tau^n \rangle.
\label{eq:genfunc}
\end{equation}
\DIFaddbegin \DIFadd{Moments of the FPT distribution can thus be computed even when it is not easy to invert $\tilde F(s)$ to obtain the real-time FPT distribution $F(t)$. }\DIFaddend For the decoration in Fig.~\ref{fig:schematic}c, \DIFdelbegin \DIFdel{$\tilde F(s)$ }\DIFdelend \DIFaddbegin \DIFadd{$\tilde F_{dec}(s)$ }\DIFaddend is a function of $\mu, a, b, k^{+}$, and $k^{-}$. For the line of states in Fig.~\ref{fig:schematic}d,  \DIFdelbegin \DIFdel{$\tilde F(s)$ }\DIFdelend \DIFaddbegin \DIFadd{$\tilde F_{line}(s)$ }\DIFaddend is a function of $k^{+}, k^{-}$, and two unknown rates $\eta^{+}, \eta^{-}$. By setting the mean and variance of \DIFdelbegin \DIFdel{these two FPT distributions }\DIFdelend \DIFaddbegin \DIFadd{$F_{dec}$ and $F_{line}$ }\DIFaddend equal to one another, we obtain analytical expressions for $\eta^{+}, \eta^{-}$ in terms $\mu, a, b, k^{+}$, and $k^{-}$. These expressions are algebraically complicated, so we do not reproduce them here; for the smallest motif considered - a triangle as depicted in Fig.~\ref{fig:schematic}c - the full effective rates are given as an example in Appendix \ref{app:eff-rates}.  By calculating effective rates for all of the decorations in a network, we can construct a single-cycle network that we expect to have a similar period $T$ (\DIFaddbegin \DIFadd{roughly }\DIFaddend captured by the first moment of the FPT distribution) and coherence $\R$ (\DIFaddbegin \DIFadd{roughly }\DIFaddend captured by the second moment of the FPT distribution) as the decorated network.

We note that this procedure does not always produce reasonable coarse-grained rates. The effective rates diverge at a value of $\mu$ that decreases as the size of the decoration increases (Table~\ref{tab:rates}). As the size of the decoration becomes larger, it will be able to support coherent oscillations of its own, leading to a system with multiple interacting periods of oscillation; in that case, one cycle is no longer dominant in terms of the dynamics of the system and we do not expect to be able to simply coarse-grain out these competing cycles. \DIFdelbegin \DIFdel{We therefore restrict our study to small cycles with 6 sides or fewer where positive effective rates are possible for values of $\mu$ of at least $\mu = 0.05$. }\DIFdelend The effective rates can also become negative if the rates in the main cycle ($k^{+/-}$) and the decoration ($a/b$) are very (orders of magnitude) different.
\DIFdelbegin \DIFdel{In the remainder of the paper, we focus on }\DIFdelend \DIFaddbegin 

\DIFadd{We therefore restrict our study to small cycles with 6 sides or fewer where positive effective rates are possible for values of $\mu$ of at least $\mu = 0.05$, and to }\DIFaddend cases where $a = k^{-}$ and $b =  k^{+}$, which we call ``cis" because the rates in the main cycle and decoration favor current in the same direction through their shared edge, or where $a = k^{+}$ and $b =  k^{-}$, which we call ``trans" because the rates in the main cycle and decoration favor current in opposite direction through their shared edge. The relative magnitude of the rates in the decoration compared to the main cycle is tuned by changing $\mu$. The \DIFdelbegin \DIFdel{cis }\DIFdelend \DIFaddbegin {\it \DIFadd{cis}} \DIFaddend configuration favors cycling in the small decoration compared to the \DIFdelbegin \DIFdel{trans }\DIFdelend \DIFaddbegin {\it \DIFadd{trans}} \DIFaddend configuration, as shown schematically in Fig.~\ref{fig:effrates}. In Table~\ref{tab:rates} we show the effective rates in the \DIFdelbegin \DIFdel{cis }\DIFdelend \DIFaddbegin {\it \DIFadd{cis}} \DIFaddend configuration in the limit where $k^{-}/k^{+} \to 0$ (for the \DIFdelbegin \DIFdel{trans }\DIFdelend \DIFaddbegin {\it \DIFadd{trans}} \DIFaddend configuration this limit simple gives $\eta^+ = k^+$ and $\eta^- = 0$). % In this limit, we find $\eta^-/\eta^+ = (x+1)(x/2 + 1)$ where $x$ is the number of states in the small cycle that are not part of the large cycle (i.e. $x = 1$ for a triangle, $x = 2$ for a square, etc.). 
 In Fig.~\ref{fig:effrates} we show $\eta^\pm$ as a function of $\mu$ for both configurations. \DIFdelbegin \DIFdel{In the cisconfiguration, $\eta^-$ quickly becomes larger than $\eta^+$ as $\mu$ is increased, creating a bottleneck in }\DIFdelend \DIFaddbegin \DIFadd{The }{\it \DIFadd{cis}} \DIFadd{configuration leads to much more dramatic changes in the effective rates than }\DIFaddend the \DIFdelbegin \DIFdel{network as the system become trapped in the decoration}\DIFdelend \DIFaddbegin {\it \DIFadd{trans}} \DIFadd{configuration (Fig.~\ref{fig:effrates}), and as a result $\R$ and $T$ change by only a small fraction as $\mu$ in turned on in networks with }{\it \DIFadd{trans}}\DIFadd{-configured decorations. For the results in the following sections we therefore focus on decorations with }{\it \DIFadd{cis}} \DIFadd{rates}\DIFaddend .

\begin{figure}
\centering
\includegraphics[width= \linewidth]{fig-2}
 \caption{Effective rates resulting from the coarse-graining procedure as a function of $\mu$, which controls the probability of entering the triangle. We consider the cases where $b = k^+$ and $a = k^-$ (\DIFaddbeginFL {\it \DIFaddendFL ``trans"\DIFaddbeginFL }\DIFaddendFL , a-b) and where $b = k^-$ and $a = k^+$ (\DIFaddbeginFL {\it \DIFaddendFL ``cis"\DIFaddbeginFL }\DIFaddendFL , c-f).  a-d: relative values of the effective rates $\eta^\pm /k^\pm$ compared to the $\mu = 0$ values. The rates change much more dramatically in the \DIFaddbeginFL {\it \DIFaddendFL cis\DIFaddbeginFL } \DIFaddendFL case, because an extra, likely path for hopping CCW through the decoration is added. e-f: For the \DIFaddbeginFL {\it \DIFaddendFL cis\DIFaddbeginFL } \DIFaddendFL case, the absolute value of the effective rates and comparison with the high-affinity expressions given in table \ref{tab:rates} (black dashed lines).}
\label{fig:effrates}
\end{figure}

\begin{table}
\centering
\begin{tabular}{|c|c|c|c|} \hline
Shape &  Exclusive Vertices  & $\eta^+$ & $\eta^-$ \\ \hline
triangle & 1 &  $k^+/(1-\mu)$ & $3k^+\mu /(1-\mu) $\\ \hline
square & 2 & $k^+/(1-3\mu)$ & $6k^+\mu /(1-3\mu)$ \\ \hline
pentagon & 3 & $k^+/(1-6\mu)$ & $10k^+\mu /(1-6\mu) $\\ \hline
hexagon & 4 & $k^+/(1-10\mu)$ & $15k^+\mu /(1-10\mu)$\\ \hline
general & x & $\frac{2k^+}{2 - x(x+1)\mu}$ &$ \frac{k^+(x+1)(1 + x/2)}{2 - x(x+1)\mu}$ \\ \hline
\end{tabular}
\caption{Effective rates across decorations with increasing number of sides, in the limit of $k^-/k^+ \to 0$. We deduce by inspection that the effective rates for a decoration with $x$ vertices that do not belong to the large cycle are: $\eta^+ = k^+/(1 - \alpha \mu)$, $\eta^- = (x + 1 + \alpha) k^+/(1 -\alpha \mu) $ where $\alpha = \sum_{i=1}^x i = x (x +1)/2$, as shown in the last row.}
\label{tab:rates}
\end{table}

\section{Predicting timescales in multicyclic networks}\label{timescales}

We can now compare $T$ and $\R$ for our coarse-grained networks to the full networks.  \DIFdelbegin \DIFdel{The }%DIFDELCMD < {\it %%%
\DIFdel{cis}%DIFDELCMD < } %%%
\DIFdel{configuration leads to more dramatic changes in the effective rates than the trans configuration (Fig.~\ref{fig:effrates}), and we found that as a result $\R$ and $T$ change by only a small fraction as $\mu$ is turned on in networks with }%DIFDELCMD < {\it %%%
\DIFdel{trans}%DIFDELCMD < }%%%
\DIFdel{-configured decorations. For the results in this section we therefore focus on decorations with }%DIFDELCMD < {\it %%%
\DIFdel{cis}%DIFDELCMD < } %%%
\DIFdel{rates. }\DIFdelend We calculate $T$ and $\R$ in three ways: first, by numerically diagonalizing the transition rate matrix of a network with explicit decorations ($\R/T_{exact}$); second, by numerically diagonalizing the transition rate matrix of the corresponding coarse-grained network ($\R/T_{CG}$), and third, using the theoretical expressions in Eqs.~\ref{eq:phinew} - \ref{eq:gamma} with the rates in the coarse-grained network as input ($\R/T_{th}$).

In the following sections we test the ability of our coarse-graining scheme combined with our analytical theory to predict $T$ and $\R$ in networks with increasing amounts of randomness. Since our theory does not contain information about the locations of the effective rates in the network, if we are able to predict these observables it means that they are insensitive to the locations of decorations on the network and will be robust to any changes in the locations of the decorations.  %We first test this prediction, and then introduce randomness to see how oscillator timescales are affected. %This would suggest that networks with many different arrangements of the same set of decorations, or of similar sets drawn from a common distribution, will be able to 

\subsection{Networks with symmetrically distributed decorations}

\begin{figure}
\centering
\includegraphics[width=\linewidth]{fig-3}
 \caption{Period $T$ and number of coherent oscillations $\R$ in networks with $N = 100$ states in the main cycle and triangle decorations with the \DIFaddbeginFL {\it \DIFaddendFL cis\DIFaddbeginFL } \DIFaddendFL configuration of rates ($a = k^-, b = k^+$). We set $k^- = 1$ and set $k^+ = \exp(\aff_0/N)$. $\aff_0$, $R_0$ and $T_0$ are the respective values in a network with no decorations. ``Exact" results are calculated from numerical diagonalization of the full network. ``CG" results are calculated from numerical diagonalization of a single cycle with $N = 100$ states with effective rates through the links where decorations are located in the full network. ``Theory" results are calculated from the theoretical expression in Eqs.~\ref{eq:phinew} \DIFdelbeginFL \DIFdelFL{- }\DIFdelendFL \DIFaddbeginFL \DIFaddFL{and }\DIFaddendFL \ref{eq:c1sq} with the rates $\{h_j\}$ given by the effective rates. On the left we show results for networks with one triangle decoration, as a function of $\mu$, which governs the probability of entering the decoration (see Fig.~\ref{fig:schematic}). On the right, we show results for networks with $\mu = 0.2$ fixed and with $m$ triangle decorations separated by $\left \lfloor{N/m}\right \rfloor$ edges. This value jumps when $N\%m = 0$, resulting in the observed discontinuous changes at these values when $\aff_0/N = 0.5$.}
\label{fig:r-t}
\end{figure}

First we test the accuracy of the coarse-grained and theoretical approximations for a fixed network topology. In Fig.~\ref{fig:r-t} we show results for networks with a large cycle of size \DIFdelbegin \DIFdel{$N=100$ }\DIFdelend \DIFaddbegin \DIFadd{$N = 100$ }\DIFaddend with a single triangle decoration as a function of $\mu$, and with $m$ evenly spaced triangle decorations as a function of $m$. \DIFaddbegin \DIFadd{All of the rates in the large cycle are set to $k^- = 1$ and $k^+ = \exp(\aff/N)$, and the rates in the small cycle are $a = k^-$ and $b = k^+$. }\DIFaddend When $m = 1$, $T/\R_{CG}$ approaches $T/\R_{exact}$ as the affinity increases, with perfect agreement in the limit of very high affinity ($k^-/k^+ = \exp(-10) \approx 0$). Note that in this limit the effective reverse hopping rates along edges representing coarse-grained decorations are not suppressed; rather they are enhanced (Fig.~\ref{fig:effrates}) and $\eta^-/\eta^+ > 1$ (Fig.~\ref{fig:effrates}f), so that even in the limit $k^-/k^+ \to 0$ it is non-trivial to predict the period.  The agreement between $T/\R_{th}$ \DIFaddbegin \DIFadd{and $T/\R_{CG}$ }\DIFaddend is also excellent. The net effect is convergence between exact values and theoretical predictions for $T$ and $\R$ with increasing affinity.

The distance between decorations is given by $\left \lfloor{N/m}\right \rfloor$. At low affinity, this results in discontinuous jumps in the values of $T/\R_{exact}$ and $T/\R_{CG}$ (yellow lines in right column of Fig.~\ref{fig:r-t}) at values of $m$ where the $N$ is an integer multiple of $m$, because the distance between the decorations is important.  At high affinity the distance no longer matters, as predicted by our theory, and the CG and exact lines become smooth and ultimately match the theory prediction.

\subsection{Networks with randomly distributed decorations}

\begin{figure}
\centering
\includegraphics[width=\linewidth]{fig-4}
 \caption{Period $T$ and number of coherent oscillations $\R$ in networks with $N = 100$ states in the main cycle and $m = 25$ decorations whose shape and location are randomly generated, with the constraint that no vertex is connected to more than three neighbors, i.e. the decorations are separated by at least one edge. The rates are chosen as in Fig.~\ref{fig:r-t}, except that we chose $\mu = 0.05$ since the effective rates for larger decorations diverge at decreasing values of $\mu$ (see Table~\ref{tab:rates}).  Each point in the scatter plots represents a specific realization of quenched disorder of shapes and locations. As $\aff_0/N = \log(k^+/k^-)$ increases, the exact, coarse-grained and theory values converge. All values are normalized by the largest value in the data on each scatter plot so that data for different affinities can be shown on the same plot.}
\label{fig:r-t-random}
\end{figure}

The success of our theory in predicting time scales in Fig.~\ref{fig:r-t} suggests that networks with many different arrangements of the same set of decorations, or of similar sets drawn from a common distribution, \DIFdelbegin \DIFdel{will }\DIFdelend \DIFaddbegin \DIFadd{can }\DIFaddend have the same values of $T$ and $\R$. We now introduce disorder by fixing the number of decorations $m = 25$ and the value of $\mu = 0.05$ and selecting the shapes and locations of decorations in the network randomly \DIFdelbegin \DIFdel{. }\DIFdelend \DIFaddbegin \DIFadd{(with all other parameters the same as in Fig.~\ref{fig:r-t}). }\DIFaddend Because our coarse-graining scheme takes \DIFdelbegin \DIFdel{in to }\DIFdelend \DIFaddbegin \DIFadd{into }\DIFaddend account the edges CW and CCW from the decoration as shown in Fig.~\ref{fig:schematic}b, we do not expect it to work well when two decorations are connected to the same vertex, and we constrain the random locations so that this does not happen (i.e. decorations are separated by at least one edge). The decorations \DIFdelbegin \DIFdel{may have between }\DIFdelend \DIFaddbegin \DIFadd{have }\DIFaddend 3 \DIFdelbegin \DIFdel{and }\DIFdelend \DIFaddbegin \DIFadd{- }\DIFaddend 6 sides. Scatter plots in which each point represents one realization of the quenched shape and location disorder are shown in Fig.~\ref{fig:r-t-random}. The results show that in these disordered networks, the exact, CG, and theory results converge at high affinity, confirming that the arrangement of decorations is unimportant at high affinity.

\subsection{Combining rate disorder and topological disorder}

\begin{figure}
\centering
\includegraphics[width=\linewidth]{fig-5}
 \caption{Period $T$ and number of coherent oscillations $\R$ in networks with $N = 500$ states in the main cycle and $m = 50$ decorations. Here, disorder in the rates of the main cycle as well as disorder in the topology of the network are considered. The shapes and locations of the decorations \DIFdelbeginFL \DIFdelFL{are randomly generated, with }\DIFdelendFL \DIFaddbeginFL \DIFaddFL{as well as many of }\DIFaddendFL the \DIFdelbeginFL \DIFdelFL{constraint that the decorations are separated by at least two edges. Then, }\DIFdelendFL rates \DIFdelbeginFL \DIFdelFL{along edges in the main cycle which }\DIFdelendFL are \DIFdelbeginFL \DIFdelFL{not connected to a vertex which is part of a decoration are }\DIFdelendFL randomly \DIFdelbeginFL \DIFdelFL{chosen from a Gaussian distribution with mean $k^+_0$ and standard deviation $0.4k^+_0$ and a lower cutoff at $0.1k^+_0$, and the locations of the decorations were randomly selected with the constraint that no node is connected to more than 3 neighbors (i.e. no decorations right next to one another). We set $a = k^- = 1$ and $b = k^+_0 = \exp(\aff_0/N)$. We introduced disorder }\DIFdelendFL \DIFaddbeginFL \DIFaddFL{generated as described }\DIFaddendFL in the \DIFdelbeginFL \DIFdelFL{probability of entering the decorations by choosing $\mu$ randomly from a uniform distribution $[0, 0.95\mu_{max}]$, where $\mu_{max}$ is the value of $\mu$ at which the effective rates in Table~\ref{tab:rates} diverge, which depends on the shape of the decoration}\DIFdelendFL \DIFaddbeginFL \DIFaddFL{text}\DIFaddendFL . Each point in the scatter plots represents a specific realization of quenched disorder of shapes, locations, and rates. As \DIFdelbeginFL \DIFdelFL{$\aff_0/N = \log(k^+/k^-)_0$ }\DIFdelendFL \DIFaddbeginFL \DIFaddFL{$\aff/N = \langle k^+/k^-\rangle$ }\DIFaddendFL increases, the exact, coarse-grained and theory values converge. All values are normalized by the largest value in the data on each scatter plot so that data for different affinities can be shown on the same plot.}
\label{fig:r-t-mix}
\end{figure}

Finally, we test how our theory performs when disorder in the rates, explored in Ref.~\citenum{DelJunco2018b}, is combined with random network topology (Fig.~\ref{fig:r-t-mix}). We now generate networks with large cycles of size $N = 500$ and $m = 50$ decorations with random shapes and random locations. The locations are again constrained so that no two decorations are connected to the same vertex. The \DIFaddbegin \DIFadd{CCW }\DIFaddend rates in the large cycle \DIFaddbegin \DIFadd{are set to $k^- = 1$, and the CW rates }\DIFaddend that are not part of the coarse-grained motif (i.e., are not connected to a vertex which is part of a decoration) are then chosen randomly \DIFdelbegin \DIFdel{, as are the values of $\mu$}\DIFdelend \DIFaddbegin \DIFadd{from a Gaussian distribution with mean $k^+_0 = \mathcal A/N$ and standard deviation $0.4k^+_0$ and a lower cutoff at $0.1k^+_0$}\DIFaddend . We set the rates in the motifs to the \DIFdelbegin \DIFdel{cis configuration}\DIFdelend \DIFaddbegin {\it \DIFadd{cis}} \DIFadd{configuration: $a = 1, b = k^+_0$, and we introduced disorder in the probability of entering the decorations by choosing $\mu$ randomly from a uniform distribution $[0, 0.95\mu_{max}]$, where $\mu_{max}$ is the value of $\mu$ at which the effective rates in Table~\ref{tab:rates} diverge, which depends on the size of the decoration}\DIFaddend . Once again, the agreement between both levels of approximation and exact results for $T$ and $\R$ is excellent in the limit of high affinity (\DIFdelbegin \DIFdel{$k^+/k^-  = \exp(10)$}\DIFdelend \DIFaddbegin \DIFadd{$\aff/N  = \exp(10)$}\DIFaddend ), with very good agreement already at moderate values of the affinity (\DIFdelbegin \DIFdel{$k^+/k^-  = \exp(2)$}\DIFdelend \DIFaddbegin \DIFadd{$\aff/N  = \exp(2)$}\DIFaddend ). 

\section{Discussion}\label{disc}

\begin{figure}
\centering
\includegraphics[width=0.9\linewidth]{fig-6}
 \caption{Compensating for changes in period by tuning the parameter $\mu$ as a function of $\Delta\aff$ in a network with a large cycle of $N = 100$ states with $m = 20$ triangle decorations symmetrically distributed on the network, with the \DIFaddbeginFL {\it \DIFaddendFL cis\DIFaddbeginFL } \DIFaddendFL configuration ($a = k^- = 1, b = k^+ = \exp(\aff/N) = \exp((\aff_0 + \Delta\aff)/N)$). $T_0$ is the period at $\aff_0/N$ and $\mu_0 = 0.5$. Dashed lines are the period as the affinity is changed with $\mu = \mu_0$; solid lines are the period when $\mu$ is changed as a function of the affinity: $\mu =\mu_0 + \Delta\mu$ with $\Delta\mu = \kappa_{comp}\Delta\aff$, as described in the text. This linear compensation mechanism is effective over a wide range of affinities $\aff_0$, compensating for changes in the period of $\sim 50\%$. Compensation occurs because by increasing (decreasing) $\mu$ the system is encouraged to spend more (less) time in the triangle decorations as the affinity is increased (decreased). Inset: The integrated deviation from perfect compensation ($T/T_0 = 1$) over the range $\Delta\aff \in [-0.3, 0.3]$ as a function of the affinity in the unperturbed network.  The mechanism is most effective when $\aff_0/N \approx 2$ or greater.}
\label{fig:compensation}
\end{figure}

%Our theory for one cycle implied that with a given distribution of rates the period of the oscillator is more tightly distributed about the average when the affinity is higher, suggesting that oscillators that consume more energy will be more robust to fluctuations in the rates. Here we extended our theory to multiple cycles. 

The results of Fig.~\ref{fig:r-t-mix} show that the time scales of an oscillator with multiple cycles and randomly distributed rates does not depend on the arrangement of these rates and cycles.  As a result, these observables can be accurately predicted from our theory with information about the probability distributions of the rates and decorations, and notably without information about the spatial arrangement of the specific network. This extends the conclusions of Ref.~\DIFdelbegin \DIFdel{\mbox{%DIFAUXCMD
\cite{DelJunco2018b} }%DIFAUXCMD
}\DIFdelend \DIFaddbegin \DIFadd{\mbox{%DIFAUXCMD
\citenum{DelJunco2018b} }%DIFAUXCMD
}\DIFaddend to the case of network topology. The motivation for studying multiple cycles is that the small cycles can represent deviations from or noise in the oscillator's average limit cycle~\cite{Pittayakanchit2018, Marsland2019}. The quenched disorder in Figs.~\ref{fig:r-t-random} and \ref{fig:r-t-mix} are meant to represent different realizations of the pathways sampled by the oscillator over time\DIFdelbegin \DIFdel{. }\DIFdelend \DIFaddbegin \DIFadd{, or by multiple copies of the same oscillator, e.g. in different cells. }\DIFaddend In this context, our results show how an oscillator whose sampled paths and rates are fluctuating over time can use a high chemical affinity to maintain a predictable and robust period.

%We think of these deviations as random fluctuations decaying over time scales shorter than the period of the oscillator, so that the topology of the multi-cycle oscillator can be slightly different at each oscillation, but will have on average the same number and distribution of small cycles.


%We think of these deviations as random fluctuations decaying over time scales shorter than the period of the oscillator, so that the topology of the multi-cycle oscillator can be slightly different at each oscillation, but will have on average the same number and distribution of small cycles. In this scenario, it is important that the oscillator be able to maintain a constant period with different realizations of quenched disorder of rates and decorations. The ability of our theory to accurately predict time scales in the full and coarse-grained networks means that the period is insensitive to the arrangement of the decorations when the affinity is high. Our work therefore suggests that high affinity can make a biochemical oscillator insensitive to fluctuations in both rates and topology, as shown in Fig.~\ref{fig:r-t-mix}. This extends the conclusions of Ref.~\cite{DelJunco2018b} to the case of network topology. if the network topology fluctuates so that an oscillator has a different number of small cycles at each 

\DIFaddbegin \DIFadd{In all of these cases, we considered disorder that maintained the average value of the rates and affinity constant. }\DIFaddend We now briefly turn our attention \DIFdelbegin \DIFdel{from local, rapidly decaying fluctuations to global , long-lived fluctuations ; e.g. }\DIFdelend \DIFaddbegin \DIFadd{to global fluctuations in the rates that result in }\DIFaddend a change in the affinity\DIFdelbegin \DIFdel{due to }\DIFdelend \DIFaddbegin \DIFadd{; for example, }\DIFaddend a shift in the overall ATP to ADP ratio in a cell \DIFdelbegin \DIFdel{, caused for instance }\DIFdelend \DIFaddbegin \DIFadd{caused }\DIFaddend by a change in light levels or a change in temperature. Biochemical oscillators often have the ability to maintain a constant period in the presence of a certain range of these changes, a feature known as input compensation~\cite{Johnson2011, Francois2012, Paijmans2017}. For a given network topology and arrangement of relative rate magnitudes, changing the affinity effectively multiplies all of the rates by a constant since $k^+/k^- \propto \exp(\aff/N) = \exp(\aff_0/N) \exp(\Delta\aff/N)$. Any change in the affinity therefore results in a change in the period. However, if the rates or the decorations in the network are allowed to vary in a manner that is coupled to the change in affinity, the oscillator may be able to absorb changes in the affinity.  Specifically, if the current increases on the network in response to an increase in the affinity, the system can increase the path length by increasing the probability of entering and remaining in small cycles.  This mechanism for compensation is a stochastic version of one that has previously been explored in deterministic limit cycles by several authors~\cite{Francois2012, Hatakeyama2015}: if an input changes the angular velocity of the limit cycle, the radius of the limit cycle must also change in response to the input in order to maintain a constant period. 
\DIFdelbegin \DIFdel{Similarly, in our multi-cycle networks, the length of the path that the system takes to complete one full oscillation is tuned by the arrangement and rates of decorations in the network.
}\DIFdelend 

We illustrate an example of this in Fig.~\ref{fig:compensation}. We choose the rates in the network as in Fig.~\ref{fig:r-t}: $a = k^- = 1$, $b = k^+ = \exp(\aff/N)$. First we hold $\mu$ fixed and vary the affinity so that the rates become $b = k^+ = \exp(\aff_0 + \Delta\aff/N)$, and show that the period changes significantly with small changes in the affinity (solid lines in Fig.~\ref{fig:compensation}) - for instance, changing the affinity from $\aff_0/N = 5$ to $\aff/N = 5.3$ reduces the period by 25\%. Then, we allow the value of $\mu$ to be appropriately coupled to the affinity and show that these changes in the period can be compensated for, reducing the change to less than 5\%. The parameter $\mu$ controls the probability of accessing the smaller secondary cycles. Hence, the parameter $\mu$ effectively controls the size of sampled orbits in our networks. 

In order to choose how $\mu$ should \DIFdelbegin \DIFdel{be }\DIFdelend depend on the affinity, we consider the Taylor expansion of the period as a function of $\aff$ and $\mu$:
\begin{align}
T(\aff, \mu) = &T(\aff_0, \mu_0)  \\
&+  \left(\frac{\partial T(\aff_0, \mu_0) }{\partial \mu}\right) \Delta\mu +  \left(\frac{\partial T(\aff_0, \mu_0) }{\partial \aff}\right) \Delta\aff \notag \\
& + \mathcal O((\Delta\mu)^2, (\Delta\aff)^2, \Delta\mu\Delta\aff) + \cdots \notag
\label{eq:texpand}
\end{align}
Perfect compensation then requires $T(\aff, \mu) = T(\aff_0, \mu_0)$ or 
\begin{equation}
0 =  \left(\frac{\partial T(\aff_0, \mu_0) }{\partial \mu}\right) \Delta\mu +  \left(\frac{\partial T(\aff_0, \mu_0) }{\partial \aff}\right) \Delta\aff + \cdots
\label{eq:texpand}
\end{equation}
In general this leads to a very complicated function of $\Delta\mu(\Delta\aff)$ with as many parameters as the Taylor expansion has terms. However, in Fig.~\ref{fig:compensation} we find numerically that over wide ranges of change in the period, it is in fact a linear function of $\Delta\aff$ and $\Delta\mu$, so that we can achieve compensation just by setting 
\begin{align}
\Delta\mu &= - \left[\left(\frac{\partial T(\aff_0, \mu_0) }{\partial \aff}\right)\bigg/\left(\frac{\partial T(\aff_0, \mu_0) }{\partial \mu}\right) \right]\Delta\aff\\
 &\equiv \kappa_{comp} \Delta\aff.
\end{align}


In the inset in Fig.~\ref{fig:compensation}, we see that this `linear compensation' mechanism works best above a minimum value of the affinity around $\aff_0/N = 1$, indicating that a high chemical affinity can support simple mechanisms for compensation.
%In Fig.~\ref{fig:compensation} we look at absolute changes in the value of the affinity and compare relative changes in the period at different $\aff_0$, so it may not seem surprising that the at a higher $\aff_0$ we find improved compensation. However, we argue that this is a meaningful comparison, since the affinity tracks the free energy of reactions in the network: $\exp(\aff_0/N) = k^+/k^- \propto [ATP]_0/[ADP]_0$, so that a small, absolute change in affinity corresponds to a fractional change in the ATP/ADP ratio:
%\begin{align}
%(1 \pm \Delta)\frac{[ATP]_0}{[ADP]_0}& \propto (1 \pm \Delta)\left(\frac{k^+}{k^-}\right)\notag \\
%&= (1 \pm \Delta)\exp(\aff_0/N)\notag\\
%& \approx \exp(\aff_0/N \pm \Delta).
%\end{align}
Indeed, using a linear approximation of our theory in Eqs.~\ref{eq:phinew} - \ref{eq:gamma}, we find that as long as the effective rates in the coarse-grained link are proportional to $k^+$ (as they are in our case; see Table~\ref{tab:rates}), all second-order and higher terms in Eq.~\ref{eq:texpand} vanish at high affinity and for large main cycle size $N$. High affinity therefore makes it easy to design (or evolve) a network of this kind with compensation, since only one parameter needs to be set, which is easily computed from the unperturbed ($\mu_0, \aff_0$) network.

Here we have illustrated compensation using $\mu$ for simplicity since it is a continuous variable. However, the number of decorations or the size of the decorations could also be used to adjust the period, since these all affect the path length of an oscillation, or alternately, the amount the time the system spends effectuating futile cycles in decorations. \DIFdelbegin \DIFdel{This, and related additional mechanisms for compensation will be explored in future work.
}\DIFdelend %DIF > This, and related additional mechanisms for compensation will be explored in future work.
%These discrete variables could still be used to finely tune the period if the network is large enough. 
%This mechanism for compensation is in some sense a stochastic version of one that has previously been explored in deterministic limit cycles by several authors~\cite{Francois2012, Hatakeyama2015}. The gist of it is that if an input changes the angular velocity of the limit cycle, the radius of the limit cycle must also change in response to the input in order to maintain a constant period. Similarly, in our multi-cycle networks, the length of the path that the system takes to complete one full oscillation is tuned by the arrangement and rates of decorations in the network.

\section{Conclusion}\label{conc}

In this paper we presented an analytical theory for computing the period of oscillations in Markov models consisting of one large cycle of size $N$ decorated with \DIFdelbegin \DIFdel{a fraction of }\DIFdelend smaller secondary cycles that are driven out of equilibrium by an affinity $\mathcal A$ (Fig.~\ref{fig:schematic}). First, we mapped the decorations on to single links that retain the mean and variance of the first passage time across the decoration. Performing this procedure for all of the decorations in the network yields a single-cycle network for which we have previously derived \DIFdelbegin \DIFdel{an analytical expression }\DIFdelend \DIFaddbegin \DIFadd{analytical expressions }\DIFaddend for the period and coherence of oscillations. Importantly, \DIFdelbegin \DIFdel{the analytical expression takes }\DIFdelend \DIFaddbegin \DIFadd{these analytical expressions take }\DIFaddend as input the rates along each edge in the network, but \DIFdelbegin \DIFdel{does }\DIFdelend \DIFaddbegin \DIFadd{do }\DIFaddend not know about their relative placement. Numerical calculations of the period at high affinity agree well with this analytical prediction (Figs.~\ref{fig:r-t} - \ref{fig:r-t-mix}). Our main result is that the ability of our theory to accurately predict the period and coherence implies that high energy dissipation makes these observables insensitive to many parameters; specifically, the arrangement of the cycles and rates in the network. As a result, oscillators represented by the models studied here can have time scales that are robust to fluctuations in rates and topology\DIFdelbegin \DIFdel{that decay over time scales comparable to the period}\DIFdelend . Finally, we showed how multi-cycle network topologies can also be exploited to achieve compensation to \DIFdelbegin \DIFdel{long-lived, global rate fluctuations due to }\DIFdelend changes in affinity, by tuning the amount of time that the system spends in the small cycles.



\begin{acknowledgments}

Thanks to Kabir Husain for generative discussions and for explaining the method to calculate first passage time distributions, and to Mike Rust for helpful discussions. We wish to acknowledge constructive comments from anonymous reviewers of Ref.~\DIFdelbegin \DIFdel{\mbox{%DIFAUXCMD
\cite{DelJunco2018b}}%DIFAUXCMD
}\DIFdelend \DIFaddbegin \DIFadd{\mbox{%DIFAUXCMD
\citenum{DelJunco2018b}}%DIFAUXCMD
}\DIFaddend , which partially motivated this work, and specifically the reviewer who suggested the scatter plot presentation of data used in Figs.~\ref{fig:r-t} - \ref{fig:r-t-mix}. CdJ acknowledges the support of the Natural Sciences and Engineering Research Council of Canada (NSERC). CdJ a \'et\'e financ\'ee par le Conseil de recherches en sciences naturelles et en g\'enie du Canada (CRSNG). This work was partially supported by the University of Chicago Materials Research Science and Engineering Center (MRSEC), which is funded by the National Science Foundation under award number DMR-1420709.  SV also acknowledges support from the Sloan Fellowship and the University of Chicago.

\end{acknowledgments}


\appendix

\section{Theory from Ref.~\citenum{DelJunco2018b}}\label{app:theory}

Our theory in \DIFaddbegin \DIFadd{Ref.~\mbox{%DIFAUXCMD
\citenum{DelJunco2018b} }%DIFAUXCMD
}\DIFaddend uses a transfer matrix formulation of the eigenvalue equation for the transition rate matrix of a single-cycle network of size $N$ \DIFaddbegin \DIFadd{where all but one set of rates are the same}\DIFaddend :
\begin{equation}
\begin{bmatrix}
f_1 \\ f_2
\end{bmatrix}
=
\MA\MB^{N-1}
\begin{bmatrix}
f_1 \\ f_2
\end{bmatrix}
\label{eq:transfer0}
\end{equation}
where $f_i$ are eigenvector elements, $\MB$ is a transfer matrix mapping eigenvector magnitudes about links with `uniform rates' $k^\pm$, and $\MA$ is a transfer matrix mapping eigenvector magnitudes about \DIFdelbegin \DIFdel{links }\DIFdelend \DIFaddbegin \DIFadd{the link }\DIFaddend with `defect rates' $h^\pm$.  $\MA$ and $\MB$ are functions of the eigenvalue $\phi$ of the transition rate matrix. By Eq.~\ref{eq:transfer0}, $\MA\MB^{N-1}$ must have an eigenvalue of one. We postulate $\phi = \phi^{(0)} + C \gamma$ with $C$ given in Eq.~\ref{eq:c}. Therefore, Eq.~\ref{eq:transfer0} is a self-consistent equation for $\gamma$ which we solve as described in Ref.~\citenum{DelJunco2018b} to obtain Eq.~\ref{eq:gamma}. Eq.~\ref{eq:transfer0} \DIFdelbegin \DIFdel{formula easily }\DIFdelend \DIFaddbegin \DIFadd{is easily extended }\DIFaddend to cases where there is more than one set of defect rates; further details can be found in \DIFdelbegin \DIFdel{the Supplementary Material of }\DIFdelend Ref.~\citenum{DelJunco2018b}. 

In Ref.~\citenum{DelJunco2018b} we approximated the product of transfer matrices as:
\begin{align}
\MA  \MB^{N-1} &\approx \MA\left[ \left(\beta_1^{(1)}\right)^{N-1} \MX_1^{(0)} + \left(\beta_2^{(1)}\right)^{N-1} \MX_2^{(0)} \right] \\
& \approx
\left(\beta_1^{(1)}\right)^{N-1}\MA \MX_1^{(0)}
\label{eq:transfer}
\end{align}
where $\beta_1^{(1)} = \exp(2\pi i/N) (1 + \gamma) =\beta_1^{(0)}(1 + \gamma)  $ and $\beta_2^{(1)} = (k^-/k^+)\exp(2\pi i/N) (1 + \gamma) = \beta_2^{(0)}(1 + \gamma) $ are first-order perturbed eigenvalues of $\MB$, and \DIFdelbegin \DIFdel{$\MX_i^{(0)} = |i\rangle \langle i|$ }\DIFdelend \DIFaddbegin \DIFadd{$\MX_i^{(0)} = |i^{(0)}\rangle \langle i^{(0)} |$ }\DIFaddend is the outer product of the $i$th unperturbed eigenvectors. In the second line we have assumed high affinity : $k^-/k^+ \ll 1$. The $i$th left and right eigenvectors of $\MB$ are given by:
\begin{align}
\langle i | =  \{ -1/\beta_j, 1\}/c_i^2 && | i \rangle = \{ \beta_i, 1\}/c_i^2
\end{align}
where $c_i$ is a normalization constant. Therefore, by using $\MX_1^{(0)}$, our theory ignored important terms containing $\gamma$. In cases where $k^-/k^+ \ll 1$ and $h^-/h^+ \ll 1$, we find that these terms cancel and our theory works with $\MX_1^{(0)}$, explaining the success of our theory in predicting timescales in Ref.~\citenum{DelJunco2018b}. However, in the coarse-grained networks studied in this paper\DIFdelbegin \DIFdel{we typically find }\DIFdelend \DIFaddbegin \DIFadd{, specifically for decorations with the $cis$ configuration, we often have }\DIFaddend $h^-/h^+ > 1$.  We therefore replace 
\begin{align}
\MX_1^{(0)} & =
\frac{1}{c_1^2}   \begin{bmatrix} 
  -\beta_1^{(0)}/\beta_2^{(0)} & \beta_1^{(0)} \\
 -1/\beta_2^{(0)} & 1
\end{bmatrix}\\
& \to
\MX_1^{(1)} =
\frac{1}{c_1^2}   \begin{bmatrix} 
  -\beta_1^{(1)}/\beta_2^{(1)} & \beta_1^{(1)} \\
 -1/\beta_2^{(1)} & 1
\end{bmatrix}
\end{align}
in Eq.~\ref{eq:transfer}, and proceed with the calculation as described in the Supplementary Material of Ref.~\citenum{DelJunco2018b}, to obtain Eqs.~\ref{eq:phinew} - \ref{eq:gamma}, where
\begin{widetext}
\begin{align}
\zeta_j &=  \frac{{h_j^-} {k^+}+{h_j^+} {k^-}-{k^-} {k^+} + 2 \gamma  {h_j^+} {k^-} +\gamma ^2 \left({h_j^+} {k^-} + {k^-} {k^+} \right) + (\gamma +1) {k^+} e^{\frac{2 i \pi }{N}} (-{h_j^-}-{h_j^+}+{k^-}+{k^+})-\left((\gamma +1) {k^+} e^{\frac{2 i \pi }{N}}\right)^2}{(\gamma +1) {h_j^+} \left({k^-}-{k^+} e^{\frac{4 i \pi }{N}}\right)} \label{eq:zeta} \\
C &=  c_1^2 k^- e^{-2\pi i/N} \label{eq:c} \\
c_1^2 & = 1-(k^+/k^-)e^{4\pi i/N}. \label{eq:c1sq}
\end{align}
\end{widetext}

\section{Calculating the first passage time distribution}\label{app:fpt}

The method for calculating the first passage time between two states is to sum over all of the paths of all lengths connecting the two states.  First we write down the FPT distribution between two connected states $1$ and $2$. If the system enters state $1$ at time $t_0$, the probability that it hops to state $j$ at time $t_1 = t_0 + \mu  t$ is:

\begin{equation}
Q_{12}(\mu  t)  = P_{12}(\mu  t) \prod_{i \neq 1, 2}\left( 1 - \int_0^{\mu  t} dt P_{1i}(t) \right)
\end{equation}
Where $P_{12}(t) = k_{12}\exp(-k_{12}t)$ is the waiting time distribution for hopping from state 1 to 2. The first term is the probability of hopping at exactly time $t_1$, while the term in parentheses is the probability that the system has not hopped to any other state in the meantime\DIFdelbegin \DIFdel{, because of course it can no longer make the transition $1 \to 2$ at time $t_1$ if it's no longer in state 1. }\DIFdelend \DIFaddbegin \DIFadd{. }\DIFaddend The net waiting time distribution out of state 1 is just the sum over connected states: $\sum_j Q_{1j}$. 

The probability of observing a particular trajectory with transitions \DIFdelbegin \DIFdel{$\{1\to 2, 2 \to 3, \cdots n-1\to n\}$ }\DIFdelend \DIFaddbegin \DIFadd{$\{1\to 2, 2 \to 3, \cdots, n-1\to n\}$ }\DIFaddend occurring at times $\{t_1, t_2, t_3, \cdots, t_{n-1}\}$ is:
\begin{equation}
Q_{12}(t_1)Q_{23}(t_2-t_1)\cdots Q_{n-1, n}(t_{n-1} - t_{n-2}).
\end{equation}
Note that state 1 is the first state that the system visits, it is not necessarily a state with a fixed label. In other words, state 1 and state 3 could both be the same state $i$, if the system jumps back to state $i$ after leaving it. State $n$ is the only state that can be visited only once, since it is an absorbing state. 

To obtain the first passage time distribution $F(n, t_{n-1} | 1, 0)$ we sum over all trajectories that start at state 1 at time $t_0$ and arrive, for the first time, at state $n$ at time $t_{n-1}$. We integrate over all possible combinations of transition times $\{t_1, \cdots t_{n-2}\}$ constrained such that $t_{n-1}$ is fixed, and sum over all paths that the system can take:
\begin{align}
&F(n, t_{n-1}| 1, 0) =  \\
&\sum_{paths} \int dt_1 \cdots dt_{n-2} Q_{12}(t_1) \cdots Q_{n-1, n}(t_{n-1} - t_{n-2}). \notag
\end{align}
We can turn this convolution \DIFdelbegin \DIFdel{in to }\DIFdelend \DIFaddbegin \DIFadd{into }\DIFaddend a product by taking the Laplace transform:
\begin{equation}
\hat Q_{ij} = \int_0^\infty dt e^{-st} P_{ij} (t)  \prod_{k \neq i, j}\left( 1 - \int_0^{t} dt' P_{ik}(t') \right)
\end{equation}
so that we have
\begin{equation}
\hat F(n, t_{n-1}| 1, 0) =\sum_{paths} \hat Q_{12} \times \hat Q_{13}\DIFaddbegin \DIFadd{\times  }\DIFaddend \cdots\DIFaddbegin \DIFadd{\times }\DIFaddend \hat Q_{n-1, n}.
\end{equation}
Since each transition in our Markov model is a Poisson process, we plug in an exponential form for $P$, giving:
\begin{align}
&\hat Q_{ij} (s) = \\
&k_{ij} \int_0^\infty dt e^{-st} e^{-k_{ij}t} \prod_{k \neq i, j}\left( 1 - k_{ik}\int_0^{t} dt' e^{-k_{ik}t'} \right) \\
& =  \frac{k_{ij}}{s+\sum_{k \neq i}k_{ik}}.
\end{align}
To sum over paths we will construct a matrix $\mathbf K$ with elements 
\begin{equation}
K_{ij} = \left\{ \begin{matrix}  \hat Q_{ij} (s) & \text{if states i and j are connected}  \\   0 &  \text{otherwise} \end{matrix} \right.
\end{equation}
$K_{1n}$ then gives us the waiting time distributions for all paths of length 1 from state 1 to $n$. $[K^2]_{1n}$ gives us paths of length 2, and $[K^m]_{1n}$ gives all paths of length $m$. Summing,
\begin{align}
\hat F(n, t_{n-1}| 1, 0) &= 1 + K_{1n} + [K^2]_{1n} + \cdots  \\
&= \sum_{m=0}^\infty [K^m]_{1n} \\
&= [(\mathbbm 1  - \mathbf K)^{-1}]_{1n}.
\end{align}
All elements of the matrix \DIFaddbegin \DIFadd{$\mathbf K$ }\DIFaddend are strictly less than 1 since $s$ is always positive, so the Frobenius norm of the matrix $\lim_{m\to \infty}\mathbf K^m <1$ and the series converges. We can then either invert the Laplace transform to obtain the FPT distribution $F(t)$, or if that's not tractable, we can obtain the moments of the distribution using Eq.~\ref{eq:genfunc}, which is what we do in this paper.


\begin{widetext}

\section{Effective rates for a triangle decoration}\label{app:eff-rates}

\begin{align}
d \eta^+ &= ({k^-}+{k^+})^2 \left(a^2 \mu  +{k^+} (a+b)\right)^2\\
d \eta^-  &=  \mu^3 a^3 b  {k^+}^2 \notag \\ 
&+  \mu  ^2 a b \left[a b ({k^-}+{k^+})^2+a {k^+} \left(-{k^-}^2-2 {k^-} {k^+}+{k^+}^2\right)+{k^+}^3 (b+{k^-}+{k^+})\right] \notag \\ 
&+\mu \Bigl[ a^3 {k^-} ({k^-}+{k^+})^2-a^2 ({k^-}+{k^+})^2 ({k^+} ({k^-}+{k^+})-b {k^-}) \notag \\
& +a {k^+} \left(b^2 ({k^-}+{k^+})^2-b {k^-} {k^+} ({k^-}+2 {k^+})+{k^+}^2 ({k^-}+{k^+})^2\right) \notag \\ 
&+b {k^+} \left(b^2 ({k^-}+{k^+})^2+b {k^+}^3+{k^+}^3 ({k^-}+{k^+})\right)\Bigr] \notag \\ 
& +{k^-} {k^+} (a+b)^2 ({k^-}+{k^+})^2 \\
d &= \mu^3 a^3 b  {k^-}\notag \\
&+a \mu  ^2 \left[a^2 ({k^-}+{k^+})^2+a b \left({k^-}^2+4 {k^-} {k^+}+{k^+}^2\right)+b {k^+} (b {k^-}-{k^+} ({k^-}+{k^+}))\right] \notag \\ 
&+\mu   \Bigl[ a^3 ({k^-}+{k^+})^2+a^2 ({k^-}+{k^+})^2 (b+{k^-}+2 {k^+}) \notag \\
&-a {k^+} \left({k^+} ({k^-}+{k^+})^2-b (2 {k^-}+{k^+}) ({k^-}+2 {k^+})\right)+b {k^+}^2 (b {k^-}-{k^+} ({k^-}+{k^+}))\Bigr] \notag \\ 
&+{k^+} (a+b)^2 ({k^-}+{k^+})^2
\end{align}
\end{widetext}

\end{document}